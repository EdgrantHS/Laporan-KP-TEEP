%
% Halaman Abstrak
%
% @author  Andreas Febrian
% @version 1.00
%

\chapter*{Abstrak}

\vspace*{0.2cm}
{
	\setlength{\parindent}{0pt}
	
	\begin{tabular}{@{}l l p{10cm}}
		Nama&: & \penulis \\
		Program Studi&: & Teknik Komputer \\
		Judul&: & \judul \\
	\end{tabular}

	\bigskip
	\bigskip

    Laporan ini merangkum kegiatan kerja praktik di \namaLab, \namaUniv, yang berfokus pada tiga proyek utama. Pertama, pengembangan \textit{WiFi Adapter} untuk mengintegrasikan data jaringan WiFi dunia nyata ke dalam kerangka \textit{Service Management and Orchestration} (SMO) O-RAN melalui antarmuka O1. Adapter ini menggunakan SNMP dan \textit{web scraping} untuk pengumpulan data dan memformatnya sesuai standar 3GPP dan O-RAN. Kedua, pembuatan \textit{RF Digital Twin} menggunakan simulator \textit{ray-tracing} NVIDIA Sionna untuk menghasilkan data \textit{fingeprint} sintetis untuk lokalisasi dalam ruangan. Proyek ini melibatkan kalibrasi parameter lingkungan 3D menggunakan optimisasi hibrida (CMA-ES dan Adam) untuk mencocokkan data dunia nyata, yang kemudian digunakan untuk melatih model lokalisasi K-Nearest Neighbors (KNN).


    \vspace{1em}
    \noindent
    \textbf{Kata Kunci:} \textit{Digital Twin}, \textit{O-RAN}, \textit{O1 Interface}, \textit{Service Management and Orchestration}, \textit{VES}, \textit{WiFi RAN}
}

\newpage
%-----------------------------------------------------------------------------%
\chapter{\babEmpat}

%-----------------------------------------------------------------------------%
\section{Kegiatan Kerja Praktik \textit{Probation}}
%-----------------------------------------------------------------------------%

Lorem ipsum dolor sit amet, consectetur adipiscing elit. Duis viverra mauris 
augue, et porta diam hendrerit efficitur. Nullam interdum nisl nec diam 
malesuada semper sed id nunc. Etiam et lacus mi. Donec eu leo ornare, 
sollicitudin justo in, malesuada libero. Phasellus pellentesque urna in mi luctus 
interdum vel eget lacus. Donec pretium erat leo, at sodales tortor imperdiet in. 
Nunc venenatis lacus massa, in tincidunt urna interdum sit amet. Aliquam iaculis 
vulputate metus a dignissim. Vivamus vulputate vitae nunc a pharetra.

Donec tincidunt nisl non elit posuere pretium. Pellentesque quis sagittis velit, 
quis vulputate purus. Quisque maximus justo posuere tempus dictum. Vivamus 
malesuada, leo eu porta elementum, purus elit tempor massa, in venenatis ligula 
mauris eget tortor. Phasellus venenatis nisi condimentum, tincidunt sem id, 
lobortis sem. Quisque volutpat, neque id ultricies finibus, ligula tortor 
condimentum nunc, ut placerat turpis libero eu dui. Donec faucibus blandit 
mauris quis posuere. Praesent a justo dapibus, blandit elit ac, pulvinar nulla. 

  \subsection{Pengenalan terhadap topik 5G dan O-RAN}

  Lorem ipsum dolor sit amet, consectetur adipiscing elit. Duis viverra mauris 
  augue, et porta diam hendrerit efficitur. Nullam interdum nisl nec diam 
  malesuada semper sed id nunc. Etiam et lacus mi. Donec eu leo ornare, 
  sollicitudin justo in, malesuada libero. Phasellus pellentesque urna in mi luctus 
  interdum vel eget lacus. Donec pretium erat leo, at sodales tortor imperdiet in. 
  Nunc venenatis lacus massa, in tincidunt urna interdum sit amet. Aliquam iaculis 
  vulputate metus a dignissim. Vivamus vulputate vitae nunc a pharetra.
  Anda diminta untuk menuliskan judul laporan, nama, npm, dan hal-hal lain yang 
  dibutuhkan untuk pembuatan template. 

    \subsubsection{Pengenalan terhadap Perkembangan 5G}

    Lorem ipsum dolor sit amet, consectetur adipiscing elit. Duis viverra mauris 
    augue, et porta diam hendrerit efficitur. Nullam interdum nisl nec diam 
    malesuada semper sed id nunc. Etiam et lacus mi. Donec eu leo ornare, 
    sollicitudin justo in, malesuada libero. Phasellus pellentesque urna in mi luctus 
    interdum vel eget lacus. Donec pretium erat leo, at sodales tortor imperdiet in. 
    Nunc venenatis lacus massa, in tincidunt urna interdum sit amet. Aliquam iaculis 
    vulputate metus a dignissim. Vivamus vulputate vitae nunc a pharetra.
    Anda diminta untuk menuliskan judul laporan, nama, npm, dan hal-hal lain yang 
    dibutuhkan untuk pembuatan template. 

    \subsubsection{Pengenalan terhadap O-RAN}

    Lorem ipsum dolor sit amet, consectetur adipiscing elit. Duis viverra mauris 
    augue, et porta diam hendrerit efficitur. Nullam interdum nisl nec diam 
    malesuada semper sed id nunc. Etiam et lacus mi. Donec eu leo ornare, 
    sollicitudin justo in, malesuada libero. Phasellus pellentesque urna in mi luctus 
    interdum vel eget lacus. Donec pretium erat leo, at sodales tortor imperdiet in. 
    Nunc venenatis lacus massa, in tincidunt urna interdum sit amet. Aliquam iaculis 
    vulputate metus a dignissim. Vivamus vulputate vitae nunc a pharetra.
    Anda diminta untuk menuliskan judul laporan, nama, npm, dan hal-hal lain yang 
    dibutuhkan untuk pembuatan template. 

    \subsubsection{Pengenalan terhadap O1 Interface}

    Lorem ipsum dolor sit amet, consectetur adipiscing elit. Duis viverra mauris 
    augue, et porta diam hendrerit efficitur. Nullam interdum nisl nec diam 
    malesuada semper sed id nunc. Etiam et lacus mi. Donec eu leo ornare, 
    sollicitudin justo in, malesuada libero. Phasellus pellentesque urna in mi luctus 
    interdum vel eget lacus. Donec pretium erat leo, at sodales tortor imperdiet in. 
    Nunc venenatis lacus massa, in tincidunt urna interdum sit amet. Aliquam iaculis 
    vulputate metus a dignissim. Vivamus vulputate vitae nunc a pharetra.
    Anda diminta untuk menuliskan judul laporan, nama, npm, dan hal-hal lain yang 
    dibutuhkan untuk pembuatan template. 

  \subsection{Pengenalan terhadap topik WiFi}

  Lorem ipsum dolor sit amet, consectetur adipiscing elit. Duis viverra mauris 
  augue, et porta diam hendrerit efficitur. Nullam interdum nisl nec diam 
  malesuada semper sed id nunc. Etiam et lacus mi. Donec eu leo ornare, 
  sollicitudin justo in, malesuada libero. Phasellus pellentesque urna in mi luctus 
  interdum vel eget lacus. Donec pretium erat leo, at sodales tortor imperdiet in. 
  Nunc venenatis lacus massa, in tincidunt urna interdum sit amet. Aliquam iaculis 
  vulputate metus a dignissim. Vivamus vulputate vitae nunc a pharetra.
  Anda diminta untuk menuliskan judul laporan, nama, npm, dan hal-hal lain yang 
  dibutuhkan untuk pembuatan template. 

    \subsubsection{Perkembangan WiFi Crawling Ubiquitty Unifi}

    Lorem ipsum dolor sit amet, consectetur adipiscing elit. Duis viverra mauris 
    augue, et porta diam hendrerit efficitur. Nullam interdum nisl nec diam 
    malesuada semper sed id nunc. Etiam et lacus mi. Donec eu leo ornare, 
    sollicitudin justo in, malesuada libero. Phasellus pellentesque urna in mi luctus 
    interdum vel eget lacus. Donec pretium erat leo, at sodales tortor imperdiet in. 
    Nunc venenatis lacus massa, in tincidunt urna interdum sit amet. Aliquam iaculis 
    vulputate metus a dignissim. Vivamus vulputate vitae nunc a pharetra.
    Anda diminta untuk menuliskan judul laporan, nama, npm, dan hal-hal lain yang 
    dibutuhkan untuk pembuatan template. 

    \subsubsection{Pengenalan terhadap Network Energy Saving}

    Lorem ipsum dolor sit amet, consectetur adipiscing elit. Duis viverra mauris 
    augue, et porta diam hendrerit efficitur. Nullam interdum nisl nec diam 
    malesuada semper sed id nunc. Etiam et lacus mi. Donec eu leo ornare, 
    sollicitudin justo in, malesuada libero. Phasellus pellentesque urna in mi luctus 
    interdum vel eget lacus. Donec pretium erat leo, at sodales tortor imperdiet in. 
    Nunc venenatis lacus massa, in tincidunt urna interdum sit amet. Aliquam iaculis 
    vulputate metus a dignissim. Vivamus vulputate vitae nunc a pharetra.
    Anda diminta untuk menuliskan judul laporan, nama, npm, dan hal-hal lain yang 
    dibutuhkan untuk pembuatan template. 

%-----------------------------------------------------------------------------%
\section{Kegiatan Kerja Praktik secara \textit{On-Site}}
%-----------------------------------------------------------------------------%

  \subsection{}



%-----------------------------------------------------------------------------%
\section{ist2ilah.tex}
%-----------------------------------------------------------------------------%
Berkas istilah digunakan untuk mencatat istilah-istilah yang digunakan. 
Fungsinya hanya untuk memudahkan penulisan.
Pada beberapa kasus, ada kata-kata yang harus selalu muncul dengan tercetak 
miring atau tercetak tebal. 
Dengan menjadikan kata-kata tersebut sebagai sebuah perintah \latex~tentu akan 
mempercepat dan mempermudah pengerjaan laporan. 


%-----------------------------------------------------------------------------%
\section{hype.indonesia.tex}
%-----------------------------------------------------------------------------%
Berkas ini berisi cara pemenggalan beberapa kata dalam bahasa Indonesia. 
\latex~memiliki algoritma untuk memenggal kata-kata sendiri, namun untuk 
beberapa kasus algoritma ini memenggal dengan cara yang salah. 
Untuk memperbaiki pemenggalan yang salah inilah cara pemenggalan yang benar 
ditulis dalam berkas hype.indonesia.tex.


%-----------------------------------------------------------------------------%
\section{pustaka2.tex}
%-----------------------------------------------------------------------------%
Berkas pustaka.tex berisi seluruh daftar referensi yang digunakan dalam 
laporan. 
Anda bisa membuat model daftar referensi lain dengan menggunakan bibtex.
Untuk mempelajari bibtex lebih lanjut, silahkan buka 
\url{http://www.bibtex.org/Format}. 
Untuk merujuk pada salah satu referensi yang ada, gunakan perintah \bslash 
cite, e.g. \bslash cite\{lankton2008introduction\} yang akan akan memunculkan 
\cite{edgrant2025kptemplate}
\cite{febrian2017uitemplate}


%-----------------------------------------------------------------------------%
\section{bab[1 - 6].tex}
%-----------------------------------------------------------------------------%
Berkas ini berisi isi laporan yang Anda tulis. 
Setiap nama berkas e.g. bab1.tex merepresentasikan bab dimana tulisan tersebut 
akan muncul. 
Sebagai contoh, kode dimana tulisan ini dibaut berada dalam berkas dengan nama 
bab4.tex. 
Ada enam buah berkas yang telah disiapkan untuk mengakomodir enam bab dari 
laporan Anda, diluar bab kesimpulan dan saran. 
Jika Anda tidak membutuhkan sebanyak itu, silahkan hapus kode dalam berkas 
thesis.tex yang memasukan berkas \latex~yang tidak dibutuhkan;  contohnya 
perintah \bslash include\{bab6.tex\} merupakan kode untuk memasukan berkas 
bab6.tex kedalam laporan.

%-----------------------------------------------------------------------------%
\section{Penulisan \textit{code} atau \textit{pseudocode} program}
%-----------------------------------------------------------------------------%

\subsection{\textit{Inline}}

Dengan perintah \verb|\verb|: \verb|System.out.println("Hello, World");| \\
Dengan perintah \textit{custom} \verb|\code|: \code{System.out.println("Hello, World"); }
Dengan perintah \verb|\mintinline|: \mintinline{java}{System.out.println("Hello, World"); }

\subsection{\textit{Multiline}}

Dengan perintah \verb|verbatim|: 

\begin{verbatim}	
public class HelloWorld {
    public static void main(String[] args) {
        // Prints "Hello, World" to the terminal window.
        System.out.println("Hello, World");
    }
}
\end{verbatim}

Dengan perintah \verb|minted|: Kode \ref{code:hw:minted}
\begin{listing}[H]
    \begin{minted}{python}
def binary_accuracy(y_true, y_pred):
    return K.mean(K.equal(y_true, K.round(y_pred)), axis=-1)


def categorical_accuracy(y_true, y_pred):
    return K.cast(K.equal(K.argmax(y_true, axis=-1),
                          K.argmax(y_pred, axis=-1)),
                  K.floatx())


def sparse_categorical_accuracy(y_true, y_pred):
    # reshape in case it's in shape (num_samples, 1) instead of (num_samples,)
    if K.ndim(y_true) == K.ndim(y_pred):
        y_true = K.squeeze(y_true, -1)
    # convert dense predictions to labels
    y_pred_labels = K.argmax(y_pred, axis=-1)
    y_pred_labels = K.cast(y_pred_labels, K.floatx())
    return K.cast(K.equal(y_true, y_pred_labels), K.floatx())


def top_k_categorical_accuracy(y_true, y_pred, k=5):
    return K.mean(K.in_top_k(y_pred, K.argmax(y_true, axis=-1), k), axis=-1)


def sparse_top_k_categorical_accuracy(y_true, y_pred, k=5):
    # If the shape of y_true is (num_samples, 1), flatten to (num_samples,)
    return K.mean(K.in_top_k(y_pred, K.cast(K.flatten(y_true), 'int32'), k),
                  axis=-1)
    \end{minted}
    \caption{An excerpt from keras: \url{https://github.com/keras-team/keras/blob/master/keras/metrics.py}}
    \label{code:hw:minted}
\end{listing}

Konfigurasi tampilan bisa dilakukan di \verb|uithesis.sty| dengan referensi dokumentasi di \url{https://github.com/gpoore/minted/blob/master/source/minted.pdf}
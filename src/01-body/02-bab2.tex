%-----------------------------------------------------------------------------%
\chapter{\babDua}
%-----------------------------------------------------------------------------%

%-----------------------------------------------------------------------------%
\section{Profil dan Sejarah Perusahaan}
%-----------------------------------------------------------------------------%

\subsection{Profil Perusahaan}
\namaUniv, yang umum dikenal juga sebagai Taiwan Tech, adalah salah satu institusi pendidikan di Taiwan yang berfokus pada riset dan perkembangan di bidang sains dan teknologi. Sejak didirikan pada tahun 1974, NTUST telah menjadi salah satu pusat pendidikan teknik dan teknologi utama di Taiwan, dengan visi dan misi yang kuat terhadap pengembangan ilmu pengetahuan dan penelitian yang relevan dengan perkembangan kebutuhan teknologi dan industri global.

NTUST berkolaborasi aktif dengan berbagai universitas lain dan industri dari seluruh dunia. Selain itu, NTUST juga berperan aktif dalam program-program pertukaran pendidikan internasional, seperti Taiwan Experience Education Program (TEEP), yang memberikan kesempatan kepada mahasiswa internasional untuk berpartisipasi dalam pendidikan dan penelitian yang komprehensif di Taiwan.

Sebagai sebuah kampus yang berlokasi di Taipei, NTUST menjadi lingkungan belajar yang dinamis dan mendukung inovasi, sehingga menjadi instansi baik untuk program magang ataupun kerja praktik yang berfokus pada peningkatan keterampilan teknis dan profesional di bidang teknologi.

\subsection{Sejarah Perusahaan}
\namaUniv, juga dikenal sebagai Taiwan Tech, didirikan pada tahun 1974 sebagai lembaga pendidikan tinggi pertama di Taiwan yang khusus didedikasikan untuk pendidikan teknologi dan teknik. Awalnya didirikan sebagai "National Taiwan Institute of Technology" (NTIT), universitas ini bertujuan untuk memenuhi kebutuhan Taiwan akan tenaga kerja yang terampil di bidang teknologi, seiring dengan pesatnya pertumbuhan industri di negara tersebut.

Sejak pendiriannya, NTUST telah mengalami berbagai transformasi signifikan. Pada tahun 1981, universitas ini dinaikkan statusnya menjadi universitas teknologi pertama di Taiwan dan kemudian berganti nama menjadi \namaUniv. Sejak itu, NTUST telah berkembang pesat, memperluas program akademiknya untuk mencakup berbagai bidang.

NTUST telah berperan penting dalam mendukung pengembangan teknologi dan inovasi di Taiwan, melalui penelitian yang berorientasi pada aplikasi praktis dan kemitraan erat dengan industri. Universitas ini terus beradaptasi dengan perkembangan global, menjalin kolaborasi dengan institusi pendidikan dan industri di seluruh dunia, dan memperkuat posisinya sebagai pemimpin dalam pendidikan teknik dan teknologi di Asia.

%-----------------------------------------------------------------------------%
\section{Kegiatan/Bidang Usaha Perusahaan}
%-----------------------------------------------------------------------------%

Kegiatan utama NTUST meliputi penyelenggaraan program-program akademik yang mencakup berbagai bidang seperti teknik elektro, ilmu komputer, teknik industri, dan teknologi informasi. Universitas ini menyediakan pendidikan berkualitas tinggi melalui program sarjana, magister, dan doktoral yang dirancang untuk memenuhi tuntutan industri dan perkembangan teknologi terkini.

Dalam hal penelitian, NTUST berkomitmen untuk melakukan penelitian terdepan dalam berbagai bidang teknologi yang terus berkembang, termasuk teknik elektro, ilmu komputer, teknik industri, dan manajemen teknologi. Penelitian di NTUST dilakukan di berbagai laboratorium dan pusat riset yang dilengkapi dengan fasilitas modern, mendukung kegiatan riset dasar maupun terapan yang dapat memberikan kontribusi langsung bagi kemajuan teknologi.

NTUST juga aktif dalam menjalin kerja sama dengan industri dan institusi penelitian lainnya, baik di tingkat nasional maupun internasional. Melalui kolaborasi ini, NTUST mengaplikasikan hasil penelitian dalam konteks praktis dan memberikan solusi inovatif untuk tantangan teknologi. Kegiatan ini mencakup pengembangan teknologi baru, evaluasi yang sudah ada, dan pemecahan masalah yang relevan dengan kebutuhan industri modern.

Selain itu, NTUST berperan dalam pengembangan sumber daya manusia melalui program pelatihan dan pendidikan yang bertujuan untuk meningkatkan keterampilan dan kompetensi mahasiswa serta profesional di bidang teknologi. Dengan pendekatan yang berorientasi pada kualitas dan inovasi, NTUST berkontribusi pada kemajuan ilmu pengetahuan dan teknologi serta memenuhi peran penting dalam pendidikan tinggi di Taiwan dan kawasan Asia-Pasifik.

%-----------------------------------------------------------------------------%
\section{Lokasi Perusahaan}
%-----------------------------------------------------------------------------%

Program TEEP diadakan di \namaUniv \ , dilaksanakan di gedung CEECS (College of Electrical Engineering and Computer Science) yang berlokasi di No. 43, Section 4, Keelung Road, Da'an District, Taipei City, Taiwan 106. Lokasi ini ada di pusat kota Taipei dengan akses transportasi yang dekat dengan berbagai fasilitas pendukung.

% \begin{figure}[H]
%     \centering
%     \includegraphics[width=0.6\textwidth]{assets/pics/lokasi.png}
%     \caption{Peta Lokasi NTUST}
%     \label{fig:lokasi_ntust}
% \end{figure}

% \begin{figure}[H]
%     \centering
%     \includegraphics[width=0.4\textwidth]{assets/pics/bmw.png}
%     \caption{Logo \namaLab}
%     \label{fig:logo_bmw}
% \end{figure}

%-----------------------------------------------------------------------------%
\section{Struktur Organisasi Perusahaan}
%-----------------------------------------------------------------------------%

College of Electrical Engineering and Computer Science (CEECS) di NTUST didirikan pada Agustus 1998 untuk mendidik mahasiswa dalam profesi di bidang teknik listrik, teknik elektronik, dan ilmu komputer. Berkomitmen untuk menjadi lembaga penelitian yang diakui secara internasional, CEECS telah mengejar pendidikan dengan kurikulum yang ketat dan penelitian tentang teknologi inovatif.

CEECS terdiri dari tiga departemen utama yaitu Departemen Teknik Elektronik dan Komputer, Departemen Teknik Elektro, dan Departemen Ilmu Komputer dan Teknik Informasi, serta satu lembaga pascasarjana yaitu Program Pascasarjana Teknik Elektro-Optik. Semua departemen ini menyediakan program doktoral dan magister dengan fokus penelitian yang berbeda-beda.

% \begin{figure}[H]
%     \centering
%     \includegraphics[width=0.8\textwidth]{assets/pics/ceecs.png}
%     \caption{Struktur Organisasi CEECS}
%     \label{fig:struktur_ceecs}
% \end{figure}

Departemen Teknik Elektronik dan Komputer di CEECS didirikan bersamaan dengan NTUST yaitu pada tahun 1974. Pendirian departemen ini bertujuan untuk mendidik para insinyur dan peneliti yang berkualifikasi tinggi untuk industri elektronik dan optoelektronik yang berkembang pesat. Dalam beberapa tahun terakhir, departemen ini telah mencapai keunggulan di bidang embedded system, desain chip IC, jaringan nirkabel dan broadband, komunikasi optik, serta bidang-bidang emerging technology lainnya.

%-----------------------------------------------------------------------------%
\section{Profil \namaLab}
%-----------------------------------------------------------------------------%

Laboratorium BMW adalah pusat riset yang berfokus pada inovasi dalam teknologi komunikasi dan informasi. Laboratorium ini mengembangkan solusi-solusi terdepan dalam bidang teknologi broadband, multimedia, dan wireless communication dengan mendalami topik-topik penelitian yang mencakup optimisasi WiFi, 5G Energy Saving, Open Radio Access Network (O-RAN), dan juga Control/User Plane Security.

Laboratorium ini berperan aktif dalam proyek-proyek inovatif yang melibatkan mitra industri dan akademik untuk menghadapi tantangan teknologi modern dalam era digital. Dengan fokus pada penelitian aplikatif, Lab BMW tidak hanya mengembangkan teori-teori baru tetapi juga implementasi praktis yang dapat diterapkan dalam industri telekomunikasi dan teknologi informasi.

\namaLab \ di NTUST ini menyediakan akses langsung ke sumber daya penelitian dan dukungan infrastruktur untuk penelitian dan pengembangan berkelanjutan. Laboratorium ini dilengkapi dengan fasilitas seperti server dan computing infrastructure, peralatan testing dan measurement untuk wireless communication, software development tools dan simulation environment, serta testbed untuk eksperimen jaringan yang memungkinkan validasi hasil penelitian secara real-time.

Lewat ini, \namaLab \ menciptakan lingkungan penelitian yang kolaboratif dan inovatif.
%-----------------------------------------------------------------------------%
\chapter{\babSatu}
%-----------------------------------------------------------------------------%

%-----------------------------------------------------------------------------%
\section{Latar Belakang}
%-----------------------------------------------------------------------------%
Seiring dengan meningkatnya kompleksitas jaringan nirkabel, kebutuhan akan manajemen dan optimisasi jaringan yang efisien menjadi semakin mendesak. Konsep seperti \textit{Open Radio Access Network} (O-RAN) mendorong standardisasi dan interoperabilitas antar komponen jaringan dari berbagai vendor, yang membuka peluang untuk kontrol yang lebih cerdas dan otomatis. Salah satu pendekatan inovatif untuk mengelola kompleksitas ini adalah dengan menggunakan \textit{Digital Twin}, yaitu representasi virtual dari sistem fisik yang dapat digunakan untuk simulasi, analisis, dan prediksi.

Dalam konteks WiFi Radio Access Network (RAN), sebuah \textit{Digital Twin} dapat memodelkan propagasi sinyal RF dalam lingkungan nyata, yang memungkinkan berbagai kasus penggunaan seperti optimisasi cakupan, manajemen energi, dan lokalisasi perangkat. Penelitian dan pengembangan di bidang ini menjadi salah satu fokus di Broadband Multimedia Wireless (BMW) Laboratory, National Taiwan University of Science and Technology (NTUST). Magang ini berpusat pada topik ``WiFi RAN'' dan ``WiFi Digital Twin''; penulis bertugas untuk mengembangkan komponen-komponen yang mendukung arsitektur ini, mulai dari pengumpulan data dari perangkat fisik, membuat protokol komunikasi sesuai standard internasional, hingga pembuatan model simulasi dan aplikasi berbasis kecerdasan buatan.

Proyek ini melibatkan pengembangan \textit{WiFi Adapter} yang sesuai dengan standar O-RAN O1 untuk mengumpulkan data dari infrastruktur WiFi multi-vendor, serta pengembangan aplikasi simulasi menggunakan NVIDIA Sionna untuk menciptakan \textit{Digital Twin} dari lingkungan RF. Kontribusi ini bertujuan untuk mendukung penelitian laboratorium yang lebih luas dalam manajemen jaringan nirkabel yang cerdas dan otonom.

%-----------------------------------------------------------------------------%
\section{Tujuan}
%-----------------------------------------------------------------------------%
Tujuan dari Kerja Praktik (KP) ini adalah untuk berkontribusi pada penelitian dan pengembangan yang sedang berlangsung di BMW Laboratory. Tujuan tersebut dapat dibagi menjadi tujuan umum dan tujuan khusus.

%-----------------------------------------------------------------------------%
\subsection{Umum}
%-----------------------------------------------------------------------------%

\begin{enumerate}
    \item Menyelesaikan syarat akademik untuk memperoleh gelar Teknik Komputer pada Program Studi \program, Fakultas \fakultas, Universitas Indonesia.
    \item Menerapkan pengetahuan dan keterampilan teknis yang telah diperoleh selama perkuliahan dalam konteks dunia nyata, khususnya dalam pengembangan teknologi telekomunikasi.
    \item Mengintegrasikan WiFi RAN dengan komponen komponen O-RAN lainnya dengan menggunakan protokol standar O-RAN O1.
    \item Membuat \textit{Proof of Concept} (PoC) dari \textit{Digital Twin} WiFi yang dapat digunakan untuk simulasi propagasi sinyal RF dan lokalisasi berbasis AI.
\end{enumerate}

%-----------------------------------------------------------------------------%
\subsection{Khusus}
%-----------------------------------------------------------------------------%
Secara khusus, tujuan kerja praktik ini adalah sebagai berikut:
\begin{enumerate}
    \item Mengembangkan dan mengimplementasikan \textit{WiFi Adapter} yang mampu mengumpulkan data \textit{Performance Management} (PM) dari perangkat jaringan multi-vendor (Aruba AP dan HPE Switch) menggunakan protokol SNMP, dan memformat data tersebut sesuai dengan standar antarmuka O-RAN O1.
    \item Melakukan integrasi antara \textit{WiFi Adapter} dengan \textit{Service Management and Orchestration} (SMO) untuk memastikan data dapat dikirim dan diproses dengan benar, serta melakukan perbaikan yang diperlukan pada format data dan mekanisme pelaporan.
    \item Mengembangkan aplikasi simulasi propagasi sinyal RF (\textit{RSSI Simulator}) menggunakan NVIDIA Sionna sebagai dasar dari \textit{Digital Twin}. Ini termasuk membuat model lingkungan 3D, melakukan kalibrasi parameter simulasi agar sesuai dengan data dunia nyata, dan menghasilkan dataset sidik jari (\textit{fingerprint}).
    \item Mengimplementasikan dan mengevaluasi model lokalisasi berbasis AI (seperti K-Nearest Neighbors) menggunakan data sidik jari yang dihasilkan oleh simulator untuk memvalidasi akurasi \textit{Digital Twin}.
    \item Berkontribusi pada proyek pendukung laboratorium, termasuk pemeliharaan dan pengembangan fitur baru untuk layanan sinkronisasi antara GitHub-Trello, Perkembangan website laboratorium, serta tugas-tugas administrasi dan dokumentasi lainnya.
\end{enumerate}

%-----------------------------------------------------------------------------%
\section{Waktu dan Pelaksanaan Kerja Praktek}
%-----------------------------------------------------------------------------%
Kerja Praktik ini dilaksanakan selama kurang lebih dua bulan \textit{remote} dimulai dari 4 April 2025 hingga 25 Juni 2025 dan empat bulan \textit{on-site}, dimulai dari 26 Juni 2025 hingga 23 Oktober 2025. Kegiatan magang \textit{on-site} dilaksanakan di Broadband Multimedia Wireless (BMW) Laboratory, Departemen \textit{Electronic and Computer Engineering}, National Taiwan University of Science and Technology (NTUST), Taipei, Taiwan.

%-----------------------------------------------------------------------------%
\section{Batasan Masalah}
%-----------------------------------------------------------------------------%
\begin{enumerate}
    \item Pengembangan \textit{WiFi Adapter} difokuskan pada pengumpulan data \textit{Performance Management} (PM) dari perangkat Access Point (AP) merek Aruba dan switch merek HPE melalui protokol SNMP. Integrasi dengan SMO dengan menggunakan format data standar O-RAN O1 berbasis VES dan Netconf.
    \item Proyek simulasi \textit{Digital Twin} menggunakan NVIDIA Sionna untuk memodelkan propagasi sinyal RF. Model lingkungan 3D yang digunakan adalah denah lantai gedung Dormitory 1 NTUST lantai 2.
    \item Model lokalisasi yang dikembangkan menggunakan algoritma K-Nearest Neighbors (KNN) dengan metrik RSSI sebagai fitur utama. Validasi model dilakukan dengan membandingkan hasil lokalisasi antara data sidik jari yang disimulasikan dengan data lokasi nyata.
    \item Kontribusi pada proyek sinkronisasi GitHub-Trello difokuskan pada pengembangan dan pemeliharaan fitur sinkronisasi otomatis antara kedua platform tersebut untuk manajemen proyek laboratorium.
\end{enumerate}

%-----------------------------------------------------------------------------%
\section{Sistematika Penulisan}
%-----------------------------------------------------------------------------%
\begin{itemize}
	\item \textbf{Bab 1 Pendahuluan:} Bab ini berisi latar belakang, tujuan, waktu dan tempat pelaksanaan, batasan masalah, serta sistematika penulisan laporan.
	\item \textbf{Bab 2 Profil Perusahaan:} Bab ini menjelaskan profil singkat dari BMW Laboratory, NTUST, termasuk sejarah, kegiatan, lokasi, serta struktur organisasi laboratorium.
    \item \textbf{Bab 3 Dasar Teori:} Bab ini menguraikan teori-teori yang mendasari pelaksanaan kerja praktik, termasuk konsep O-RAN, Digital Twin, protokol SNMP, serta dasar-dasar simulasi propagasi sinyal RF dan lokalisasi berbasis AI.
    \item \textbf{Bab 4 Metodologi Pelaksanaan Kerja Praktek:} Bab ini menjelaskan metode dan langkah-langkah yang diambil selama pelaksanaan kerja praktik, mulai dari pengembangan WiFi Adapter, integrasi dengan SMO, hingga pembuatan dan evaluasi model Digital Twin.
    \item \textbf{Bab 5 Hasil dan Pembahasan:} Bab ini menyajikan hasil yang diperoleh selama kerja praktik, serta analisis dan pembahasan terkait pencapaian tujuan yang telah ditetapkan.
\end{itemize}

